\documentclass[a4paper,12pt]{article}
\usepackage[utf8]{inputenc}
\usepackage[frenchb]{babel}
\usepackage{graphicx}
\usepackage[T1]{fontenc}
\usepackage{fancyhdr}
\usepackage{graphics}
\usepackage{verbatim}
\usepackage{url}
\setcounter{page}{1}
\fancyhead[L]{\leftmark}
\fancyhead[R]{}

\fancyfoot[C]{}
\fancyfoot[R]{\thepage}



\begin{document}
\begin{titlepage}
		\newpage
		\thispagestyle{empty}
		\begin{center}
			\includegraphics[scale=0.15]{images/logo_fac}	
		\end{center}
		\begin{center}
			\vspace{0.3cm}
			\large  
			\textbf{Mémoire de Master 2 MIAGE}\\
			\vspace{0.7cm} \large 
			\vskip 0.2in
		\end{center}
		\fbox{%
			\parbox{\textwidth}{%
				\begin{center}
					\begin{center}
						\large \textsc{\textbf{ Comment, à partir de la reconnaissance faciale et de la fouille de données, peut-on identifier l’ensemble des sosies potentiels d’une personne ?  }}
					\end{center}
					
				\end{center}
			}%
		}
		\begin{center}
		    Application à l’univers du cosplay
		\end{center}
		
		\begin{flushleft}
			\begin{center}
			
					\textbf{Réalisé par} \\
					\hspace{0.1cm} VILLER Nathanaëlle \\
					\textbf{Tuteur}\\
				 \hspace{0.1cm} GUEHIS Sonia \\
			
			\end{center}
		\end{flushleft}
	
		\begin{center}
			\vspace{0.1cm} \textbf{Année scolaire 2017-2018}
		\end{center}
		\begin{center}
			\includegraphics[scale=0.2]{images/LytholiaCosplay.png}	
		\end{center}
		
%\newpage
%\thispagestyle{empty}
%\section*{Remerciements}
 
\newpage
\thispagestyle{empty}
\begin{flushright}
\textbf{Remerciements} \\
Merci à ma tutrice, Sonia Guehis, à Gabriel Viller et Laurent Moysoulier pour leurs expertises, leurs conseils et leurs soutiens sur mon travail.
\end{flushright}
\newpage
\tableofcontents
\thispagestyle{empty}
\end{titlepage}

\newpage
	
%\addcontentsline{toc}{section}{Introduction}
%\section*{Introduction}

%\textit{}
%\\ \\

%\newpage
\pagenumbering{arabic}
\addcontentsline{toc}{section}{Glossaire}
\section*{Glossaire}
\begin{description}
\item [Sosie :] Personne qui a une parfaite ressemblance avec une autre au point qu'on peut les confondre.
\item [Cosplay :] Le phénomène du cosplay a commencé dans les années 40 aux États-Unis. Mais c'est lors des années 70 et 80 que le cosplay atteint son paroxysme grâce aux succès de Star Trek et Star Wars. Le Japon va ensuite adopter ce phénomène et lui donner le nom qu'il porte aujourd'hui : le cosplay (qui vient de la contraction de costume et playing). Le principe est simple : il faut se déguiser en un personnage de fiction et incarner ce personnage. Le mouvement va s'intensifier au Japon et Tokyo deviendra le centre névralgique de ce phénomène. Il faudra attendre les années 90 pour que le cosplay arrive en Europe. La scène européenne se démarquera par sa volonté de respect du craft (fabrication manuelle de son costume). Les conventions et concours vont se multiplier. Les cosplayers vont de plus en plus chercher de nouvelles techniques pour pouvoir ressembler le plus possible aux personnages qu'ils essaient d'incarner. Il y a 2 grandes catégories qui se démarquent. Ceux qui cosplayent leurs personnages préférés sans se soucier de la morphologie et ceux qui essaient de trouver des personnages qui leur correspondent physiquement.
\item [Cosplayeur :] Nom masculin, peut s'écrire aussi cosplayer, ou cosplayeuse pour les femmes. Il s'agit d'une personne pratiquant l'art du cosplay.
\item [Cosplayer :] Verbe d'action signifiant l'art de pratiquer le cosplay. 
\item [Crawler / Robot d'indexation :] Un robot d’indexation est un robot logiciel utilisé par les moteurs de recherche pour parcourir le réseau et les sites web afin d’archiver les pages web au sein des index de référencement.
\item [Morphologie :] (du grec morphé, « forme » avec le suffixe -logie, « discours ») est l'étude de la forme externe et de la structure des êtres vivants.

\end{description}

\newpage
\addcontentsline{toc}{section}{Introduction}

\section*{Introduction}
Quand on parle de sosie, on pense généralement à la personne qui pourrait être notre jumeau. Mais si nous agrandissions le champ de recherche ? Au lieu de se limiter à des êtres humains réels, si nous pouvions chercher nos sosies dans le domaine du dessin animé, des peintures, des personnages de films, etc. Cela ne serait-il pas plus intéressant ? N'êtes-vous pas curieux de savoir si un personnage fictif vous ressemble ? 
\\\\
En premier lieu nous regarderons la pertinence de ces questions. Est-ce qu'une communauté aurait besoin d'un tel outil, et qu'est-ce qu'en pense, de manière plus large le public. 
\\\\
Dans un second temps nous répondrons aux questions suivantes : qu'existe-t-il actuellement pour permettre de trouver un sosie ? Comment fonctionnent ces solutions ? Quelles sont leurs limites ? Et ainsi faire un état de l'art dans le domaine des sosies. Nous regarderons aussi ce qui existe en termes de crawler, lesquels peuvent être pertinent pour nous ? Et enfin, dans cette partie, nous aborderons les différentes techniques de comparaison d'images. 
\\\\
Dans une troisième partie sera présentée une solution architecturale permettant de répondre à notre problématique. Cette dernière sera schématisée et expliquée. Les contraintes légales de notre solution y seront aussi abordées. 
\\\\
Finalement, nous conclurons sur cette problématique, réflexion et solution, qui auront été abordées tout au long de notre recherche. 
\\\\
Alors, comment, à partir de la reconnaissance faciale et de la fouille de données, peut-on identifier l’ensemble des sosies potentiels d’une personne ?


\newpage
\section{Périmètre d'étude}
\subsection{Contexte}
Pour ce mémoire, je souhaitais me focaliser sur la partie analyse et conception de solutions nouvelles plus que des aspects de développement que je réalise déjà en entreprise. L’objectif étant de diversifier mes expériences et de proposer des pistes de développement pertinentes voir novatrices sans spécialiser mon travail sur la réalisation technique.\\ \\
Des idées me sont venues autour d’un phénomène émergent que j’observe depuis plus de dix ans dans mon environnement social et personnel. Ce phénomène concerne l’univers du déguisement, maquillage et imitation personnage fantastique communément appelés le cosplay. Ce phénomène est aussi lié aux univers cinématographiques et du jeu vidéo.
\\ \\Cette pratique est en pleine expansion, ce qui est créateur de nouveaux marchés. Les personnes impliquées étant pour la plupart des passionnés investissant beaucoup de temps et d'argent. Certains en ont même fait leur profession. L'évolution de cette branche amène de plus en plus la mise en place de solutions dédiées. Par exemple, depuis début 2010 des matériaux thermoplastiques ont été conçus pour cette communauté tel le «Worbla». Actuellement, des concurrents ont vu le jour, ce qui montre l'évolution d'un marché de niche vers un marché concurrentiel. Ce milieu semble néanmoins toujours dans sa phase d'expansion et reste dans l’attente de nouvelles solutions pour ses usagers. Preuve de sa jeunesse, mais de sa croissance et de sa reconnaissance, le mot cosplay vient juste de rentrer dans le dictionnaire Robert 2019. 
\\ \\
Mon objectif a été de rechercher autour de ce thème des idées permettant de proposer un nouvel outil pertinent. La recherche sur ce thème, pour trouver une idée neuve et innovante, fut réalisée de septembre à octobre 2017. Les idées imaginées furent vérifiées comme n’existant pas déjà et pouvant être viables. Il fallut aussi rechercher les solutions logiciels permettant d’assister ces idées dans leur réalisation. Fin octobre, l’essentiel de ce travail avait été réalisé. Les recherches furent prolongées jusqu'à la fin de l’année 2017.\\ \\
En conclusion de ce travail préparatoire  : la thématique semble  «techniquement» pertinente, mais la meilleure façon de pouvoir mesurer cette pertinence est de réaliser une enquête d'intérêt.
\subsection{Enquête d'intérêt}
\textbf{Cibles} \\
Pour que l’enquête soit pertinente, il faut s’assurer que les publics visés soient en accord avec la solution proposée. En fonction du profil de chaque public, un enjeu différent sera établi.
Il ne semble cependant pas adaptés de devoir sortir de notre enquête toute personne pouvant s’intéresser à notre solution quelques en soit la raison, puisque le but de cette enquête est justement de déterminer les intérêts spécifiques de chaque profil rencontré.
Deux catégories de publics paraissant évidente et ont donc été établis comme ceux concernés par les personnages imaginaires dans leurs loisirs et ceux qui ne le sont pas nécessairement.\\ \\
Si la première catégorie concerne évidemment le profil des cosplayeurs, les artistes créant des personnages imaginaires sont aussi un profil concerné. À l’inverse la deuxième catégorie concerne, pour sa part, uniquement le profil de toutes personnes n’appartenant pas à la première catégorie. \\
La première catégorie contient en réalité un troisième profil, celui des artistes qui sont en même temps des cosplayeurs. Les enjeux sur ce profil étant le cumul des deux autres.
\\ \\
Les différents profils pertinents définis sont donc :
\begin{itemize}
    \item Les Cosplayeurs,
    \item Les Artistes,
    \item Les Artistes-Cosplayeurs,
    \item Toutes autres personnes pouvant s'intéresser à la solution. \\
\end{itemize} 


\textbf{Contraintes} \\
L’idéal d’un sondage est d’avoir un très bon taux de réponse en un temps limité. Suivant le mode de transmission choisi, par voie numérique, il ne semble pas pertinent de laisser plus d’une à deux semaines pour le remplir. Passé ce délai, si la personne sondée ne l’a pas fait, elle ne le fera pas. Pour s’assurer que les personnes contactées se sentent concernées pour remplir le sondage, il est préférable de contacter des personnes qui me connaissent ou qui sont dans mes réseaux.
\\ 

\textbf{Stratégie de ciblage} \\
Pour toucher un échantillon le plus large possible de personne pouvant s’intéresser à la solution, et pas uniquement les cibles les plus spécialisées, le formulaire a été partagé via Facebook. Un autre intérêt de cette méthode est de pouvoir toucher un échantillon aléatoire. En effet, la publication contenant le sondage ne sera pas affichée à tous les contacts, l'algorithme de Facebook choisira à qui cette publication sera affichée. À défaut d’être exhaustif, la sélection sera pseudo-aléatoire et non contrôlé par le créateur du sondage. De plus, afin de compléter l'enquête le sondage a aussi été diffusé via mon réseau professionnel, pour atteindre des personnes de tranche d'âges plus large que celle des contacts Facebook.
Pour contacter nos trois cibles prioritaires, le sondage a été diffusé par le biais de groupes Facebook de cosplayeurs et d’artistes, mais aussi grâce à des pages publiques de ces communautés. \\

\textbf{Outil de sondage} \\
J'ai choisi de passer par Google Form car, c'est un outil simple, pratique et très bien conçu pour les personnes sondées qui ne maîtrisent pas bien l'informatique. De plus, ces derniers n'ont pas de réticence à cliquer sur un lien qui vient de chez Google et qui est envoyé par mail ou sur les réseaux sociaux, car il inspire confiance (ne fais pas peur). \\ 

\textbf{Formulaire de l'enquête}\\
Le formulaire comprenait jusqu'à 16 questions suivant les réponses. Ci-dessous est présentée la conception du séquencement de l'enquête : 
\\\\
\begin{figure}[!ht]
\centering
			\includegraphics[scale=0.25]{images/diagflow.png}
			\caption{Diagramme de séquencement de l'enquête}
\end{figure}

Voici quelques exemples de questions : 
\begin{itemize}
    \item \textit{Cosplayeurs : }Comment choisissez-vous votre nouveau cosplay ? 
    \item \textit{Cosplayeurs : }Lorsque vous choisissez votre cosplay, choisissez-vous un personnage qui vous ressemble morphologiquement ? 
    \item \textit{Artistes : }Seriez vous d'accord de fournir vos œuvres (sous format image) à l'outil pour compléter la base de données ?  
    \item \textit{Tout le monde : }Aimeriez-vous avoir la possibilité de choisir la catégorie de recherche ? 
    \item \textit{Tout le monde : }Aimeriez-vous avoir accès facilement à une liste de personnages qui vous ressemblerait ? 
    \item \textit{Tout le monde : }L'idée de cet outil vous parait-elle ? (Note de 1 à 5) \\
\end{itemize}

\textbf{Résultats} \\
La catégorie Artistes-Cosplayeurs va alimenter à la fois les catégories Artistes et Cosplayeurs pour un même sujet. L'enquête a été remplie par 487 personnes réparties en : 

\begin{figure}[!ht]
\centering
			\includegraphics[scale=0.6]{images/patate.PNG}
			\caption{Résultat du profil général des sondés }
		\end{figure}

Groupes majoritaires :
\begin{itemize}
    \item 79,1\% sont des femmes
    \item 78,4\% ont entre 18 et 30 ans
    \item 75,3\% sont des cosplayeurs \\
\end{itemize}

Le groupe le moins représenté est les plus de 30 ans avec 12,6\%.  \\

Résultats comparés : 
\begin{itemize}
    \item 85,8\% des cosplayeurs sont des femmes
    \item 87,0\% des cosplayeurs entre 18 et 30 ans sont des femmes
    \item 86,9\% des cosplayeurs, entre 18 et 30 ans, qui ont donné une notation de 4 ou 5, sont des femmes
    \item 97,0\% des cosplayeuses entre 18 et 30 ans, qui ont donné une notation de 4 ou 5, ont dit vouloir utiliser de telles fonctionnalités dans leurs recherches \\
\end{itemize}

On constate donc qu'un public spécifique se démarque. Avec de forts pourcentages, les femmes, entre 18 et 30 ans, et tout particulièrement les cosplayeuses, semblent notre cible majoritaire. Leurs intérêts pour cette solution montrent effectivement la pertinence pour ce travail de recherche.  Notre population d'origine, ayant répondu aux questionnaires, étant majoritairement des femmes, 79,1\%, on doit relativiser ce résultat (les hommes semblant sous représentés). En effet environ 73\% des hommes sondés ont manifesté leurs intérêts pour notre solution, quelsque soient leurs âges ou leurs profils (majoritairement des cosplayeurs). 
Ces chiffres étant assez proches de ceux obtenus pour les femmes, on peut supposer que l'ensemble des cosplayeurs entre 18 et 30 ans sont une cible prioritaire, quelsque soient leurs genres. 
\begin{figure}[!ht]
\centering
			\includegraphics[scale=0.8]{images/femme.PNG}
			\caption{Cosplayeurs entre 18 et 30 ans}
		\end{figure}

\textit{Les chiffres utilisés dans ce mémoire, sauf précisions, proviennent donc de cette étude. Ces résultats ont été obtenus grâce à l'utilisation d'un outil de Data Visualisation : Qlik Sense Desktop.}


\subsection{Vous avez dit sosies ?}
Lorsque nous parlons de sosie, la première image qui nous vient est une personne en tout point conforme à nos traits physiques avec qui nous pourrions échanger nos identités. Il existe de nombreuses œuvres sur ce concept par exemple le «\textit{Le Prince et le Pauvre}», roman de Mark Twain publié en 1882 qui a eu plusieurs adaptations en film, où l'on voit un prince et un pauvre physiquement identiques échanger de place. Ou bien «\textit{Deux pour le prix d'une}» roman de 1949 d'Erich Kästner qui a aussi connu des adaptations en films notamment par Disney. Le roman raconte une histoire entre 2 fillettes qui découvrent qu'en réalité elles sont sœurs jumelles et décident d'inverser leur place auprès de parents divorcés. Ou encore le drama coréen «\textit{Who Are You: School 2015}» sorti en 2015 qui raconte l'histoire de 2 sœurs, elles aussi jumelles, séparées à la naissance. Là aussi, une des sœurs va devoir usurper la place de l'autre afin de faire avancer l'intrigue. Ce genre d'histoire mêlant jumeaux et usurpation/échange d'identité est fréquente et le public en est friand. Nous voyons que ce phénomène n'est pas restreint ni à une époque ni à une culture. De plus, la recherche de sosies de personnalités connues est fréquente, exemple pour le tournage d'un film. 
\\ \\
Rechercher son sosie par curiosité, ou pour toute autre raison, même lorsque l'on n'a pas eu la chance de naître avec un jumeau est monnaie courante. Mais pourquoi se contenter de chercher uniquement parmi les êtres humains ? Et si un personnage fictif pouvait nous ressembler ? Si un personnage d'une série, d'un dessin, de jeux vidéo ou même une peinture pouvait nous ressembler ? Et si l'on pouvait analyser des photos pour récupérer tous les personnages fictifs nous ressemblant en prenant en compte le style graphique de chaque œuvre ? Aimeriez-vous avoir accès facilement à une liste de personnages qui vous ressemblent ? 82,8\%, soit environ 404 parmi les 487 personnes sondées disent que oui. 

\subsection{Cas d'application}
70,8\% des cosplayers font parfois attention à cosplayer un personnage qui a une morphologie proche de la leur. Et parmi eux, un peu moins de la moitié trouvent cela important et y font scrupuleusement attention. Bien que le cosplay soit un loisir où chacun doit se sentir libre d'incarner le héros de son choix, on voit qu'il y a une forte tendance chez les cosplayers à faire attention à la similitude entre eux et les personnages choisis. 
\\ \\
Bien qu'environ 83\% des cosplayers fonctionnent au «coup de cœur» et choisissent leur projet cosplay parmi les personnages qu'ils connaissent et leur plaisent. Ils sont tout autant à avoir répondu qu'ils seraient intéressés à accéder à une liste de personnages leur ressemblant. Ce projet n'est pas nécessaire à la pratique cosplay, mais permettrait d'ouvrir de nouveaux champs d'études. En effet la réalisation de cosplay ne se ferait plus que sur des personnages connus à l'avance, mais au contraire serait un point d'accès pour découvrir un nouvel univers. Cela se ressent dans les commentaires renseignés lors de l'étude : avoir accès à une liste de personnage leur ressemblant permettrait de \textit{"favoriser le choix de costume par rapport à notre corps, pour les rendre plus appropriés"}, mais encore que \textit{"c'est une bonne chose, cela peut permettre d'avoir plus de possibilités"} ou alors peut être \textit{"très bien pour les gens ayant du mal à faire un choix ou n'ayant pas d'idée"}, ... Donc, proposer une liste de personnage ressemblant à la photographie d'une personne serait une amélioration, dans la communauté cosplay, d'autant plus que, comme le montre l'étude, cela est très fortement demandé.  

\subsection{Problématique}
80,3\% des personnes ayant été sondées (donc pas uniquement les cosplayers) ont répondu être intéressées pour utiliser une application permettant de récupérer une liste de sosies, dans le cadre du cosplay, mais aussi du déguisement. De plus, à peine 6,2\% trouvaient l'idée mauvaise ou plutôt mauvaise.  \\\\
Pour le développement d'une telle application, nous pourrions nous baser en premier lieu sur la reconnaissance faciale, puis agrandir le champ à la reconnaissance morphologique pour la comparaison d'image et ainsi repérer les sosies. 
Afin d'avoir un ensemble de données suffisant à une telle recherche, il serait intéressant d'ajouter des logiciels de type «crawler» afin d'obtenir l'ensemble du champ d'application du web pour l'étude.
Le sujet mérite donc d'être approfondi. \\\\
\textit{Comment, à partir de la reconnaissance faciale et de la fouille de données, peut-on identifier l'ensemble des sosies potentiels d'une personne ? }
\section{État de l'art}
Dans un premier temps, il convient d'étudier les solutions existantes pouvant, en partie ou complètement, répondre à cette problématique. 
L'objectif est de trouver une solution permettant de renseigner en entrée une photographie/image et d'obtenir à partir de celle-ci, une image/photographie similaire à celle-ci. De plus, nous souhaitons pouvoir comparer la morphologie ainsi que des caractéristiques physiques tels que la couleur de peau, la couleur des yeux ou la couleur des cheveux. Et tout cela, en parcourant le plus grand nombre de résultats possibles, issus directement de la toile. Et enfin, les images en retour ne doivent pas être uniquement des photographies de personnes réelles.

\subsection{Solutions existantes et limites}
\subsubsection{Moteurs de recherches par mots-clefs}
Les moteurs de recherches par mots-clefs ne sont pas pertinents dans le cadre de notre recherche, ces derniers ne permettent pas d'avoir un retour suffisamment précis et clair, puisqu'un mot-clef ne suffit pas à définir une image tout entière, et sont déjà très étudiés et répandus. 
\subsubsection{Les moteurs de recherches d'images inversées}
Les moteurs de recherches d'images inversées permettent de mettre en entrée une image et en sortie de lister tous les sites utilisant cette image. Cela ne correspond pas à ce que nous cherchons. Mais l'algorithme utilisé pour parcourir le web à la recherche de tous les sites utilisant une image peut être intéressant, puisqu'il va recenser à travers internet tous les sites qu'il peut trouver possédant l'image en question. Cela ressemble presque à ce que ne voulons faire. 

\subsubsection{Recherche par couleur}
Des sites proposent de mettre en entrée des couleurs et en sortie de lister des images possédant les couleurs en question. En fonction des sites les images peuvent posséder d'autres couleurs ou n'ont que celles sélectionnées, par exemple sur le site labs.tineye.com/multicolr. Cela ne correspond pas non plus à notre recherche. 
\subsubsection{Google Image}
 \begin{figure}[!ht]
    \centering
        \includegraphics[scale=0.7]{images/telechargement.PNG}
        \caption{Chat de Google Images}
    \end{figure}
Google est connu pour permettre la recherche par mot-clef. Mais il permet aussi de mettre en entrée une image pour trouver les images similaires. Faisons l'expérience avec 2 images très différentes. La première provient d'une recherche par mots-clefs sur Google Image (Figure 4) et la seconde est une photographie de moi dont je suis l'auteur (Figure 5). \\ \\
   \begin{figure}[!ht]
    \centering
        \includegraphics[scale=0.04]{images/27785203_330504664108478_1945544054_o.jpg}
        \caption{Photographie de Nathanaëlle VILLER}
    \end{figure}
En utilisant l'image de chat dans la recherche de Google Image on obtient le résultat ci-dessous. (Figure 6) Nous pouvons constater qu'il retourne les sites utilisant l'image fournie et qu'il propose de chercher les images similaires.  Malheureusement, cette recherche va «traduire» en mots-clefs notre image initiale pour fournir des images peu similaires. Voir Annexe 1. \\ 

    \begin{figure}[!ht]
    \centering
        \includegraphics[scale=1]{images/ResGI.PNG}
        \caption{Résultat de la recherche via une image sur Google}
    \end{figure}

Le résultat des images similaires fourni par Google avec la photographie en Figure 5 n'est pas plus concluant non plus. (Figure 7) En effet, Google va aussi «traduire» la photographie en un mot-clef : girl, et ressort des images de femmes et d'hommes. Les photographies sont similaires dans leur style. On peut voir qu'il y a une personne sur un fond en général unique, mais en rien ces personnes ne ressemblent à Nathanaëlle VILLER.
    \begin{figure}[!ht]
    \centering
        \includegraphics[scale=0.7]{images/Res3GI.PNG}
        \caption{Résultat de la recherche des images similaires par Google pour la Figure 5}
    \end{figure}
    
\subsubsection{Sites de recherches de sosie}
Il existe quelques sites permettant de chercher son sosie. L'un des plus connus est twinstrangers.com. Ce site demande une photo, ainsi que quelques informations comme la date de naissance et le genre. Le site retourne ensuite la personne avec le plus de similitudes. Pour accéder au résultat suivant, il faut payer. 
Voir Annexe 2 pour les 3 premiers résultats obtenus à partir de la photographie en Figure 5. \\ \\
Il s'agit de jeunes filles d'un âge plus ou moins proche de Nathanaëlle, leurs yeux sont également clairs et elles portent des lunettes. Mais on ne peut pas parler de sosie. D'après les termes d'utilisations du site, nous pouvons apprendre que seul le visage (à partir de la photo) sert à faire la recherche. Cette dernière sera faite parmi la base de données du site. Rien n'est dit par rapport au genre, mais puisqu'il faut le renseigner nous pouvons nous questionner si ce dernier ne permettait pas de faire un premier filtre. Ce site a pour but de permettre à des sosies de se trouver puis de se rencontrer. Cette plateforme se limite à une base de données interne et à des personnes physiques, ce qui est différent de notre objectif.
Mais il semble intéressant d'étudier comment fonctionne l'algorithme de ce site, quelle technique il utilise pour comparer les visages. 
\\\\
Il existe d'autres sites tels que pictriev.com qui déduit si nous sommes un homme ou une femme ainsi que notre âge et nous fournis les célébrités auxquelles nous ressemblons le plus. Nous pouvons constater en annexe 3 que ce dernier n'est pas au point. Il en va de même pour celebslike.me (annexe 4). Tous les 2 sont des sites permettant de chercher son sosie parmi les personnalités connues. Encore une fois cela ne va pas avec nos objectifs. Au vu des résultats de faible correspondance, les algorithmes sont loin d'être performants.

\subsubsection{Google Arts \& Culture}
\begin{figure}[!ht]
    \centering
        \includegraphics[scale=1]{images/GAC.PNG}
        \caption{Exemple trouvé sur Twitter}
    \end{figure}
Google Arts \& Culture est une application américaine qui permet de trouver son sosie en œuvre d’art. Google Arts \& Culture a collaboré avec plus de 1 200 musées, galeries et institutions de 70 pays pour rendre leurs expositions accessibles à tous en ligne, sûrement aussi pour avoir une base de données interne à interroger. Malheureusement cette fonctionnalité «trouver son double dans une œuvre d'art» de l'application n'est disponible qu'aux États-Unis et un test n'a donc pas pu être possible. Mais nous pouvons trouver des articles qui en parlent. Le fonctionnement est simple, il suffit de charger une photographie ou un autoportrait et l'algorithme va analyser les traits du visage à plus de 70 000 portraits provenant de collection du monde entier. Twitter a été rempli de test des internautes américains dont voici un exemple bluffant (Figure 8).\\ \\
La technologie de reconnaissance faciale FaceNet développée par Google, utilisée dans cette application est désormais disponible dans une version open source officieuse, OpenFace. Cette dernière est très intéressante pour notre sujet de recherche. Les résultats obtenus par l'application sont concluants et l'on observe des similitudes entre les deux photographies. Il s'agit de la première application connue permettant de trouver son sosie avec une œuvre d'art. Ce qui rejoint fortement notre objectif bien que se limitant à des œuvres d'art.  

\subsubsection{Les sites de rencontre}
Les sites de rencontrent se mettent de plus en plus dans la recherche de sosie. L'objectif est de permettre aux internautes de chercher un sosie de leur célébrité préférée ou de leur ex-partenaire ou de la personne leur plaisant, mais étant déjà prises. Dans les sites de rencontre permettant ce genre de recherche nous pouvons citer Badoo et Match (Meetic). 
Badoo est assez frappant car les similitudes trouvées sont fortes (même si certains correspondent à de faux comptes usurpant la photo d'une autre personne). (Figure 9)
\begin{figure}[!ht]
    \centering
        \includegraphics[scale=1]{images/badoo.PNG}
        \caption{Exemple avec Badoo}
    \end{figure}
    
\subsubsection{Conclusions sur ces solutions}

Actuellement, les solutions permettent surtout de trouver un sosie de nous-mêmes avec une personne physique. Mais aucune d'entre elles n'est une solution probante. Il existe des solutions (application ou site web), mais en général elles se servent dans leur base de données locale et pas «sur internet». \\ \\
Ce qui manque à ces applications : 
\begin{itemize}
    \item Une recherche à travers internet 
    \item Une sortie de sosie qui ne soit pas uniquement une personne physique réelle 
    \item Une reconnaissance morphologique en plus de celle faciale \\
\end{itemize}
Pour ce qui est du second point, Google a développé un algorithme permettant de chercher un sosie avec une œuvre d’art de type peinture ou dessin exposé dans un musée. Il faudrait donc trouver une solution pour proposer un outil à plus large spectre, qui reprendrait l’existant, mais pour le mener plus loin. 
Ouvrir le retour de sosie à toute œuvre : peinture, dessins, coloriage, photographie … Pour trouver une personne physique ou un personnage fictif en termes de sosie. Et, effectuer la recherche en dehors d'une base de données locale, mais s'appuyer sur toutes les ressources dont dispose internet. 

\subsection{Besoins utilisateur}
Suite à notre étude il est ressorti deux grands besoins utilisateurs. Le premier étant de pouvoir trouver son sosie et l'autre de pouvoir aider à compléter la base de données. Et cela rentre bien avec l'objectif d'avoir une solution simple. La solution pourrait être proposée sur différentes plateformes : un site web et une application sur les différents supports mobiles. Pour permettre aux gens d'avoir un avant-goût de l'application et leur donner envie de l'utiliser, cette dernière pourrait être utilisée de façon réduite sans avoir besoin de s'enregistrer. \\
\\ Partir sur une base de données collaborative est envisageable car, 84\% des artistes interrogés sont d'accord de partager leurs œuvres à l'application si ces dernières servent uniquement à la recherche de sosies et qu'elles ne sont pas détournées de l'application. On peut voir cela comme une forme de partenariat: les artistes remplissent notre base de données et en échange nous leur permettons de se faire connaître. Cela permet d'avoir une seconde source de provenance pour remplir la base en plus de la fouille de données. \\
\\De plus, notre proposition de pouvoir trier sa recherche de sosie (par genre : bande dessinée, comics, manga, ...) a fort plu au vu du résultat de notre étude : 90,6\% répondent «Oui» à la question : «Aimeriez vous avoir la possibilité de choisir la catégorie de recherche ?»  

\subsection{Fouilles de données}
Pour la partie fouilles de données, nous utiliserons un algorithme de crawling web, ou en français, robot d'indexation.
Il s’agit de bots logiciels qui vont parcourir les sites internet avec
pour objectif principal de les indexer. Le parcours se fait en profondeur, le
logiciel accédant tour à tour à chaque page ou site externe grâce aux liens
hypertextes présents dans la page en cours d’analyse (exemple : lien URL) puis
réitère l’opération. On peut ajouter des paramètres au bot afin qu’il recherche
en même temps que sa «lecture». Dans notre cas, l'objectif sera de récupérer les images de certains sites ainsi que quelques informations liées à l'image (auteur, univers, etc.). \\  \\ 
On trouve facilement plusieurs crawler tels que :
\begin{itemize}
    \item \textbf{GoogleBot} est le crawler le plus connu. Il permet d’indexer les sites
internet. Il fait principalement de la «lecture» de site. Dans notre cas c'est insuffisant puisque l'on veut faire une recherche dans les sites parcourus. 
Dans le même ordre d’idée, nous pouvons retrouver les bots des moteurs
de recherche tels que BingBot, Slurp Bot (Yahoo), DuckDuckBot,
BaiduSpider (Baidu est un moteur de recherche en chinois) ou encore
YandexBot (utilisé pour les moteurs de recherches russes).
\item \textbf{Facebook External Hit}, Facebook utilise ce crawler afin de permettre
à ces utilisateurs, lorsqu’ils envoient (publient) un lien internet
à un autre utilisateur du site, de récupérer et afficher certaines
informations importantes par exemple : des extraits vidéos,
des images ou des titres.
Ce bot ne correspond pas à ce que nous souhaitons mettre en place.
Par contre il pourrait être un élément de réponse à la question des
droits d’auteurs. En effet, si l’outil était développé comment
renvoyer l’image trouvée ainsi que toutes ses informations sans
enfreindre les lois sur le droit à l’image ? Avec ce type de bot, cela est
possible car il permet de récupérer et afficher l’information choisie et
de renvoyer vers sa localisation dans la toile.
\item \textbf{HTTrack} est un logiciel aspirateur de sites internet qui crée des miroirs des sites web pour une utilisation hors ligne. Cela ne correspond pas à ce que nous cherchons et ne nous apportera pas d'aide dans notre recherche. 
\item \textbf{Alexa Crawler} (développé par Amazon) va indexer les sites après
les avoirs triés en fonctions de certains critères. Cela se rapproche de ce que nous souhaitons en apportant une dimension de classement selon des critères multiples qu’il est possible de sélectionner et filtrer.
\item \textbf{Nutch} est un crawler destiné à un moteur de recherche open source.
Son architecture est modulaire et permet de créer des plugins pour différentes phases du processus : récupération des données, analyse des documents, recherche, etc. Nutch est le crawler qui se rapproche le plus de la solution que nous
cherchons.
\end{itemize} 
\subsection{Traitement et comparaison d'images}
Ce qui est actuellement le plus courant est la comparaison faciale. La dernière technologie permettant de comparer des photographies avec des images est FaceNet. Ce dernier correspond exactement a ce que nous cherchons. 
Pour ce qui est de la morphologie, il n'existe pas encore d'algorithme efficace. Le plus simple restant la comparaison par pixel. \\ \\
En mai 2015, Benjamin Billet, ingénieur en informatique, spécialisé dans les travaux de recherche et développement (prototypage, veille technologique, industrialisation et amélioration de l'existant), a justement publié un article sur les différentes méthodes les plus connues et pratiquées pour la recherche d'image similaire. La comparaison d'image par empreinte ne permet pas d'obtenir de résultats concluants, car, elles cherchent à obtenir un identifiant par rapport aux couleurs et à leur disposition sur l'image et nullement sur des caractéristiques morphologiques. Le but étant de trouver des copies conformes entre images. \\ \\
Par contre un histogramme de couleurs en moyenne par bloc nous permettrait de comparer nos images pour trouver les plus proches : 
\begin{description}
\item [Histogramme des couleurs : ]
Un histogramme des couleurs représente la distribution des couleurs dans l'image, c'est-à-dire le nombre de pixels dont la couleur appartient à une plage donnée, les différentes plages couvrant la totalité de l'espace colorimétrique. L'histogramme nous donne donc une représentation statistique de l'image, deux images étant analysées en comparant leurs histogrammes. Cette méthode, bien que très simple à implémenter, présente toutefois le désavantage d'être purement colorimétrique : en pratique, deux images aux sujets très différents peuvent avoir des histogrammes très proches.
\item [Moyenne par bloc : ]
Cette méthode est elle aussi assez simple et consiste à diviser l'image en blocs et à calculer la moyenne colorimétrique de chaque bloc. La comparaison se fait en calculant la moyenne des différences de couleur entre les blocs de chaque image, sous réserve que les deux images aient été hashées avec le même nombre de blocs. \\
\end{description}
Actuellement, les technologies développées permettent de comparer des images dans leur ensemble, de trouver celles identiques (grâce à une empreinte), ou de trouver un visage et de le comparer. La partie sur la morphologie est encore en recherche. On peut trouver des thèses traitant le sujet, mais pas de solution concluante existante. Ainsi pour notre problématique, si nous voulons comparer la morphologie de 2 personnes/personnages il nous faudra attendre une avancée technologique. 
\\\\
\textit{Alors, comment, à partir de la reconnaissance faciale et de la fouille de données, peut-on identifier l’ensemble des sosies potentiels d’une personne ?}

\section{Solution Architecturale}
Aujourd'hui il n'existe pas de solution à notre problématique. Voici donc une proposition d'architecture qui y répond. Dans le diagramme ci-dessous, à chaque numéro (de 1 à 5) correspond une partie dans la suite de ce chapitre. (Figure 10)

\begin{figure}[!ht]
    \centering
        \includegraphics[scale=0.4, angle=-90]{images/SchemaArchitecture.png}
        \caption{Architecture de notre solution}
    \end{figure}
    
 \subsection{Fouilles périodique} 
 En utilisant un crawler, par exemple Nutch, en le configurant d'un plugin de récupération d'images et d'informations liées à celles-ci, l'algorithme va parcourir différents sites web et récupérer les différentes images trouvées. Il faudra ensuite trier les images pour ne garder que celles que nous trouvons intéressantes : celles contenant une personne/ un personnage. Pour cela on utilisera la reconnaissance faciale, par exemple FaceNet que l'on aurait configuré aussi bien pour les œuvres d'art que pour d'autres univers tels que le comics, les jeux vidéo, etc.,  ainsi l'on pourra écarter les images ne contenant aucun portrait et celles contenant plusieurs portraits. Les images conservées seront converties en un «code» permettant ensuite leurs comparaisons et seront enregistrées avec toutes sortes d'informations récupérées sur le site, comme son URL. Attention on n'enregistre pas l'image par contre, uniquement ses informations et le «code» produit par notre algorithme. Tout ceci sera enregistré dans la base de données générale de l'application. Ce «code» sera le résultat d'un traitement pouvant permettre la comparaison ensuite. Par exemple, pour faire une comparaison colorimétrique. 
 
 \subsection{Actions utilisateur}
 L'utilisateur aura 3 types d'actions possibles via l'interface. 
 \begin{itemize}
     \item \textbf{Gestion utilisateur} : Il y aura toutes les actions liées à la gestion du compte utilisateur telles que la création de comptes et la modification de ce dernier. 
     \item \textbf{Ajout d'images} : Un utilisateur pourra ajouter une image à son compte, sans cela il ne pourra pas faire de recherche de sosies. Mais il pourra aussi ajouter des images pour remplir de façon collaborative la base de données de l'application. 
     \item \textbf{Recherche de sosie} : L'utilisateur pourra faire une recherche de sosies de façon générale ou spécifique grâce à la fonctionnalité de tri permettant de chercher un sosie de en fonction d'univers précis. 
 \end{itemize}
 \subsection{Enregistrements d'images utilisateur}
 En plus de récupérer des images via un crawler sur le web, la base de données pourra être remplie de façon collaborative par les utilisateurs. Toutes les images uploadées par les utilisateurs sur l'application seront testées pour vérifier qu'il s'agit bien d'une personne grâce à la reconnaissance faciale et ensuite elles seront aussi converties en un «code» pour être enregistrées dans la base de données. Par contre, à l'inverse des images récupérées sur les sites web parcourus, ces images seront enregistrées en local dans une base de données image. Ainsi chaque «code» sauvegardé dans la base de données générale, lors d'un ajout d'image par un utilisateur, référencera une image enregistrée dans la base de données image. 
 \subsection{Gestions utilisateur}
 Les modifications que l'utilisateur souhaitera apporter en dehors du changement de sa photo seront transcrites en requête envoyée à la base de données.  Cela permettant la création, récupération, modification et suppression d'un compte. 
 \subsection{Comparaison}
 La comparaison se fait sur la photographie liée au compte de l'utilisateur. Grâce à un algorithme de comparaison d'image, par exemple la comparaison par bloc de la colorimétrie, les différents «codes» de chaque image seront comparés. On en gardera les meilleurs résultats pour ensuite les afficher. Les différents types d'algorithmes de comparaison d'images transcrivent la plupart du temps les images en un «code», c'est ce dernier qui permet la comparaison et de savoir si les images sont complémentaires et à combien de pourcentage. Pour plus d'exactitude dans la comparaison, on peut aussi comparer les visages entre eux. Grâce à la reconnaissance faciale, avec FaceNet par exemple, on pourrait calculer la complémentarité entre 2 visages. Au même titre il faudrait pouvoir comparer la morphologie de chacun. Mais pour l'heure il n'existe pas de «reconnaissance morphologique» aboutie en termes d'algorithme. 

\subsection{Affichage}
L'affichage des images enregistrées dans la base de données image pourront simplement être affichées dans notre application. Pour ce qui est des images récupérées sur la toile, on pourrait utiliser : 
\begin{itemize}
\item Le système de Facebook, permettant d'afficher une miniature de l'image et un lien cliquable vers le site et l'image.
\item Ou alors nous pourrions intégrer une page web dans notre application et ainsi intégrer la page web contenant l'image.
\item Ou encore utiliser le même affichage que Google Images, permettant d'afficher une image puis d'accéder au site la contenant. 
\end{itemize}  
\subsection{Contraintes légales}
Pour éviter les problèmes de droits d'auteurs, nous ne pouvons pas enregistrer en local les images trouvées sur les sites web. Comme notre objectif est de comparer les images, il suffit de «convertir» l'image en un code. Ce code que nous obtenons peut, lui, être enregistré en local. Pour retrouver et afficher l'image nous ne garderons d'elle que le lien URL pour la retrouver. \\ \\
Dans la législation française, le droit d'auteur permet à un auteur de choisir de quelle manière son œuvre peut être partagée, reproduite, réutilisée ... D'un auteur à l'autre, ses autorisations changent. Enregistrer une image pour la partager à un public dans le cadre de notre application peut être illégal en fonction des droits décidés par les auteurs. Par contre référencer les sites où ont été partagées ces images n'est pas une infraction. \\ \\
De plus, les articles 226-1 à 226-8 du Code civil précisent que « tout individu jouit d’un droit au respect de sa vie privée ainsi que d’un droit à l’image. En vertu de ces dispositions, il va de soi que la publication ou la reproduction d’une image (photographie) sur laquelle une personne est facilement reconnaissable n’est autorisée qu’avec son consentement préalable, et ce, que l’image soit préjudiciable ou non. Nous ne pouvons donc pas nous permettre de réutiliser des photographies librement dans notre application. \\ \\
La solution qu'utilise Facebook ou Google est donc à exploiter. Quant aux images uploadées par les utilisateurs, il suffira de leur faire accepter le droit de diffusion dans l'application de leurs images et photographies. 

\newpage
\addcontentsline{toc}{section}{Conclusion}
\section*{Conclusion}
Il existe à ce jour, aucune solution permettant de chercher son sosie parmi des personnes imaginaires. Il existe cependant des projets plus ou moins abouties et qui permettent de trouver une image similaire parmi des photographies d'êtres humains ou parmi des stars. Mais dans le domaine de l'oeuvre fictive le marché est encore jeune, la concurrence est faible et les développements peu aboutis. Pour l'instant une seule application réussie a été recensée et ne fonctionne que pour les œuvres d'art.Sponsorisé par Google, celle-ci n'a pour objectif de proposer des innovations techniques,mais pas de répondre à des besoins d'utilisateurs. Pour finir, la grande majorité des projets existants sont privés, nous ne pouvons donc pas récupérer , étudier et incorporer leur méthodologie. 
\\ \\ 
Du côté du public les retours sont positifs. Les sondés sont de nature curieuse et pensent que la solution serait intéressante. Quant à la communauté cosplay, une telle solution apporterait une aide précieuse et permettrait de développer cette dernière. Le public le plus sensible à la solution sont des femmes cosplayeuses, parfois artistes ayant un âge compris entre 18 et 30 ans. Il serait intéressant dans la suite du développement de la solution, d'intégrer certaines personnes de cette population afin de ne pas perdre de vue les besoins utilisateurs et que les fonctionnalités développées soient en phases avec ceux-ci.
\\ \\
Suite à notre réflexion et en s'appuyant sur la solution architecturale proposée, il devrait être possible de développer une application répondant à notre besoin. Et cela, en reprenant des algorithmes existants, en les coordonnant et en utilisant les interactions possibles entre eux pour qu'ils s'appliquent à notre cas. 
\\ \\ 
Pour ce qui est de la comparaison morphologique, il faudra par contre attendre un développement plus poussé des technologies et algorithmes actuels ou bien tester des solutions proposées dans certaines thèses.
\\
De récentes études liées à l'explosion des capacités des technologies de Machine Learning montrent de nouvelle façon de reconnaître une personne, de par sa posture, ou par rapport à des éléments imperceptibles à première vue par l'humain par exemple le lobe d'oreille ou la plissure du nez. Il serait donc possible dans un futur proche de couplet notre solution avec des algorithmes prédictifs issus du Machine Learning afin de rendre celle-ci plus performante sur la comparaison d'un corps humain dans sa globalité.
\newpage



\section{Annexes}
Annexe 1 : 
\begin{figure}[!ht]
    \centering
    \includegraphics[scale=0.8]{images/Res2GI.PNG}
    \caption{Résultat de la recherche des images similaires par Google pour la Figure 1}
\end{figure}

Annexe 2 : 
\begin{figure}[!ht]
    \centering
        \includegraphics[scale=0.45]{images/ResS1.PNG}
        \caption{Premier sosie}
    \end{figure}
    \begin{figure}[!ht]
    \centering
        \includegraphics[scale=0.3]{images/ResS12.PNG}
        \caption{Deuxième sosie}
    \end{figure}
\begin{figure}[!ht]
    \centering
        \includegraphics[scale=0.3]{images/ResS13.PNG}
        \caption{Troisième sosie}
    \end{figure}
    
  \newpage
Annexe 3 : 
\begin{figure}[!ht]
    \centering
        \includegraphics[scale=1]{images/ResS2.PNG}
        \caption{Résultat du second site}
    \end{figure} 
\newpage
Annexe 4 :
\begin{figure}[!ht]
    \centering
        \includegraphics[scale=1]{images/ResS3.PNG}
        \caption{Résultat du troisième site}
    \end{figure}
\newpage
Sources : 
\begin{itemize}
    \item \textbf{TinEye} — \textit{Moteur de recherche d'image} —\url{http://labs.tineye.com/multicolr/}
    \item \textbf{Google} — \textit{Moteur de recherche d'image} — \url{https://www.google.fr/}
    \item \textbf{TWIN STRANGERS} - \textit{Site de recherche de sosie} — \url{http://www.twinstrangers.com/}
    \item \textbf{PicTriev} — \textit{Site de recherche de sosie} — \url{http://www.pictriev.com}
    \item \textbf{CelebsLike.Me} — \textit{Site de recherche de sosie} — \url{http://www.celebslike.me/}
    \item \textbf{Badoo} — \textit{Site de rencontre} — \url{https://badoo.com/}
    \item \textbf{Match / Meetic} — \textit{Site de rencontre} — \url{https://www.meetic.fr/}
    \item \textbf{Numerama} - \textit{Article sur la reconnaissance faciale} - \url{https://www.numerama.com/tech/126861-openface-un-script-de-reconnaissance-} \url{faciale-open-source.html}
    \item \textbf{Medium} - \textit{Article sur la reconnaissance faciale} - \url{https://medium.com/@vinayakvarrier/building-a-real-time-face-recognition-} \url{system-using-pre-trained-facenet-model-f1a277a06947 }
    \item \textbf{IDEMIA } —\textit{Article sur la reconnaissance faciale } — \url{https://www.morpho.com/fr/reconnaissance-faciale }
    \item \textbf{OpenFace API Docs} — \textit{Documentation de OpenFace} — \url{http://openface-api.readthedocs.io/en/latest/}
    \item \textbf{u Ottawa } - \textit{Article sur le crawling} - \url{http://ssrg.site.uottawa.ca/docs/CASCON2013.pdf }
    \item \textbf{Oncrawl } - \textit{Article sur le crawling } - \url{http://fr.oncrawl.com/2016/introduction-crawler-web/}
    \item \textbf{Business to web } - \textit{Article sur le crawling } - \url{https://www.b2w.fr/1489-quest-ce-crawl.html}
    \item \textbf{Définitions marketing } — \textit{Article sur le crawling } — \url{https://www.definitions-marketing.com/definition/robot-d-indexation/ }
    \item \textbf{Keycdn } - \textit{Article sur le crawling} - \url{https://www.keycdn.com/blog/web-crawlers/}
    \item \textbf{Benjamin Billet} — \textit{Comparaison d'images} — \url{http://benjaminbillet.fr/blog/index.php?article1/detection-images-similaires }
    \item \textbf{CFC Centre Français d'exploitation du droit de Copie} — \textit{Droit d'auteur} — \url{http://www.cfcopies.com/juridique/droit-auteur }
    \item \textbf{economie.gouv.fr} — \textit{Droit d'auteur} — \url{https://www.economie.gouv.fr/files/files/directions_services/apie/propriete_intellectuelle/publications/utiliser_contenu_etapes_essentielles.pdf }
    \item \textbf{Service Public} — \textit{Droit d'auteu}r — \url{https://www.service-public.fr/professionnels-entreprises/vosdroits/F23431 }
   
    
    
\end{itemize}



%\subsection{}
%\subsubsection{}

%\begin{itemize}
%\item	
%\item	
%\item
%\item	
%\item	
%\end{itemize}


%\begin{center}
%		    \includegraphics[scale=0.6]{images/cas_utilisation}	
%		\end{center}
	


%\begin{verbatim}

\end{document}